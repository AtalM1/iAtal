\documentclass[a4paper,11pt]{article}
\usepackage[utf8]{inputenc}
\usepackage[T1]{fontenc}


%opening
\title{iAtal Manual}
\author{Agathe \textsc{Moll\'e} \and Gr\'egoire \textsc{Jadi} \and Hugo \textsc{Mougard} \and Joseph \textsc{Lark} 
\and Lo\"\i c \textsc{Jankowiak} \and No\'emi \textsc{Sala\"un} \and R\'emi \textsc{Bois}}
\date{}

\begin{document}

\maketitle


\section*{Preface}
This manual documents the use of the application iAtal.  As a
reference manual, it will describe the different features of the
application. It includes a tutorial in order to help the
beginners. You will also find an help to compile, install and run the
program.  All the specific terms will be explained in a glossary.
  %TODO : blablabla, skip chapters blabla
  
\section*{Introduction}
The application described here is named iAtal. It's a rover behaviour
simulation platform.  A rover is a robot which executes some tasks in
a self-sufficient way, and in an unknown environment.  It owns sensors
that help it to feel and understand its close environment and
actuators that allow it to move and to interact with the map.


The iAtal program can deal with any map in TMX format. Those maps can
be created with Tiled %TODO ref%)
or manually.  You can also create your own exploration strategy, your
rover's sensors and actuators, thanks to Python.  Those features will
be detailed further.

\section{How to run iAtal}

\subsection{Requirements}
% TODO : Attentes de l’application en terme d’architecture machine et
% de logiciels pré-installés

\subsection{Installing iAtal}
% TODO

\subsection{Running iAtal}
% TODO : façons correctes de lancer iAtal...
There are two different ways to run the program :
\begin{itemize}
\item Whithout any option :
      \begin{verbatim}
	.\ui
\end{verbatim}
  The program will then start empty and you will have to set the map
  and the rover strategy later. (See chapter...)
      
\item Directly with a personnalised map and/or exploration strategy :
      \begin{verbatim}
	.\ui [-h|--help] [{-m|--map} map_path] [{-s|--strategy} strat_path]
\end{verbatim}
  Description of these options :
  \begin{itemize}
  \item \verb!-h! or \verb!--help! : Produce help message
  \item \verb!-m map_path! or \verb!--map map_path! : Set the path to
    the map to use. It must be a file in TMX format (.tmx) (See
    chapter...)
  \item \verb!-s strat_path! or \verb!--strategy strat_path! : Set the
    path to the strategy to use. It must be a file in Python (.py). It
    requires a map to be set too. (See chapter...)
  \end{itemize}
\end{itemize}

\section{Importing a map}

\subsection{Creating a TMX map}

To create your own map, it is strongly recommended to use the software
Tiled (ref, doc).  Tiled generates XML files wich are on the TMX
format, supported by iAtal. However, you should know that iAtal map
loader is limited to maps with 4 layers or less.
    
Once your map created, you just have to save it, and to load it in
iAtal.




\section{Defining a robot and a strategy with Python}

\subsection{Defining a robot with robot\_init}

\subsubsection{Where the robot starts}

\subsubsection{Actuators}

\subsubsection{Sensors}

\subsubsection{Extending your robot abilities}


\subsection{Defining a strategy with strat}

\subsubsection{Using the sensors and the actuators}

\subsubsection{Some useful functions}

\subsubsection{Specifying the end}



\end{document}
