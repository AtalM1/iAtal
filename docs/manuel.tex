\documentclass[a4paper,11pt]{article}
\usepackage[utf8]{inputenc}
\usepackage[T1]{fontenc}
\usepackage{listings}
\usepackage[english]{babel}
\usepackage[autolanguage]{numprint}
\usepackage{amsmath}
\usepackage{listings}
\usepackage{xcolor}
\lstset{%
  basicstyle=\footnotesize,%
  keywordstyle=\color{blue},%
  stringstyle=\color{brown}}
\usepackage{graphicx}
\usepackage{hyperref}
\graphicspath{{./img/}}
\DeclareGraphicsExtensions{.png,.jpg}
\usepackage{libertine}

\hypersetup{
  colorlinks,
  citecolor=blue,
  filecolor=blue,
  linkcolor=blue,
  urlcolor=blue
}

\newcommand\iAtal{$i$\textsc{Atal}}

% opening
\title{\iAtal{} Manual}
\author{Agathe \textsc{Mollé}
  \and Grégoire \textsc{Jadi}
  \and Hugo \textsc{Mougard}
  \and Joseph \textsc{Lark} 
  \and Loïc \textsc{Jankowiak}
  \and Noémi \textsc{Salaün}
  \and Rémi \textsc{Bois}}
\date{\today}

\begin{document}

\maketitle

\tableofcontents



\section*{Preface}
This manual documents the use of the application \iAtal{}.  As a
reference manual, it will describe the different features of the
application. It includes a tutorial in order to help the
beginners. You will also find an help to compile, install and run the
program.  All the specific terms will be explained in a glossary.

\section*{Introduction}
The application described here is named \iAtal{}. It's a rover behaviour
simulation platform.  A rover is a robot which executes some tasks in
a self-sufficient way, and in an unknown environment.  It owns sensors
that help it to feel and understand its close environment and
actuators that allow it to move and to interact with the map.


The \iAtal{} program can deal with any map in TMX\cite{tmx}
format. Those maps can be created with Tiled\cite{tiled} or manually.
You can also create your own exploration strategy, your rover's
sensors and actuators, thanks to Python.  Those features will be
detailed further.

\section{How to run \iAtal{}}

\subsection{Requirements}
First and foremost, \iAtal{} is intended to run on an Unix-like
system. It uses extensively linux-centric libraries such as gtkmm and
the glib.

You can try to install it and run it on Windows since the GTK+
libraries have been ported back there, but no support is provided in
this manual or anywhere else in the documentation to do so.

Be aware of \iAtal{} dependencies. In the listing below, the dev
suffixes are only needed if you compile the application. To run it,
the standard packages are fine enough.

The following listing gives the package names for a Debian setup (so
Ubuntu is good to go too). They should be similar enough in the
distribution you use.

\begin{itemize}
\item the GTK+ library for C++ version 3.0 or higher with the headers
  for development. (\texttt{gtkmm-3.0-dev})
\item the Boost library (\texttt{libboost-all-dev})
\item CppUnit to run the unit tests
\item Doxygen to generate the documentation
\end{itemize}

\subsection{Installing \iAtal{}}
The install process is GNU compliant:
\begin{lstlisting}[language=sh]
$ sh autogen.sh
$ ./configure
$ make
\end{lstlisting}

The \texttt{configure} script can in particular be called with an
option to use python 3.2 instead of python 2.7 if you prefer the latter:

\begin{lstlisting}[language=sh]
$ ./configure --with-python-3.2
\end{lstlisting}

\subsection{Running \iAtal{}}
There are two different ways to run the program :
\begin{itemize}
\item Whithout any option :
\begin{lstlisting}[language=sh]
$ ./iatal
\end{lstlisting}
  The program will then start empty and you will have to set the map
  and the rover strategy later.
  
\item Directly with a personnalised map and/or exploration strategy :
\begin{lstlisting}[language=sh]
$ ./iatal [-h|--help] [{-m|--map} map_path] [{-s|--strategy} strat_path]
\end{lstlisting}
  Description of these options :
  \begin{itemize}
  \item \verb!-h! or \verb!--help! : Produce help message
  \item \verb!-m map_path! or \verb!--map map_path! : Set the path to
    the map to use. It must be a file in TMX format (.tmx)
  \item \verb!-s strat_path! or \verb!--strategy strat_path! : Set the
    path to the strategy to use. It must be a file in Python (.py). It
    requires a map to be set too.
  \end{itemize}
\end{itemize}

\section{Importing a map}

\subsection{Creating a TMX map}

To create your own map, it is strongly recommended to use the software
Tiled\cite{tiled}.  Tiled generates XML files wich are on the TMX
format, supported by \iAtal{}. However, you should know that \iAtal{} map
loader is limited to maps with 3 layers. Those layers are :
\begin{itemize}
 \item ground
 \item obstacles
 \item objects
\end{itemize}
\label{avaibleLayers}

\iAtal{} only understands TMX maps with separated tilesets. The tilesets
are described in TSX files (.tsx). It is required that both the
tileset and the layers have the same name (the three names above).

\iAtal{} ignores all the layers or tilesets that are not named
accordingly to the convention above. It also ignores object groups and
recognizes 1 property per tile only. That means that above 1 property
per tile, \iAtal{} has undefined behaviour: it's not clear which one
it will chose.

The \iAtal{} has released a TMX parsing library that will allow
\iAtal{} to remove some of those limitations (if not all). Though, due
to time constraints, the current release is quite strict about what it
expects as far as TMX maps are concerned.

\subsection{Loading your map}

Once your map is created, you just have to save it, and to load it in
\iAtal{}, through the GUI or thanks to the \textsc{-m} option of the
CLI.
 By default,
the application is given with a map called map.tmx, that you can
find in the directory src/tmx/resources.

\paragraph{}
The map loading is necessary to run the program. Of course, if you
already gave the map path to iAtal thanks to the running options,
you don't need to repeat it through the loading button.


\section{Defining a robot and a strategy with Python}

This section will describe how to define a robot and a strategy to
apply to the selected map. The user will have to implement some
functions in order to describe how the robot will behave and some of
its features. These functions have to be located in a file which will
be given to the application. The table \ref{tab:PyMandatoryFunc}
presents the mandatory functions to use in the strategy file. You will
also need to import all from the classes module. You will also need
elements and enums modules. Your imports should be at least :

\begin{figure}[h]
  \begin{center}
    \begin{lstlisting}[language=Python]
      import elements
      import enums
      from classes import *
    \end{lstlisting}
    \caption{the mandatory imports}
  \end{center}
\end{figure}

\begin{table}[h]
  \begin{center}
    \begin{tabular}{|c|p{10cm}|}
      \hline
      Name        & Purpose                      \\
      \hline
      robot\_init & Sets the start position, the direction faced by the
      robot and the sprites used to represent it \\
      \hline
      strat       & Describes an iteration of the strategy used. Sensors and
      actuators are used here. The robot may move or perform some
      actions in each iteration.                 \\
      \hline
      isEnded     & Function returning True when the strategy is finished
      (the robot completed its job) or False if more iterations are to
      come.                                      \\ 
      \hline 
    \end{tabular}
    \caption{\label{tab:PyMandatoryFunc} The mandatory functions}
  \end{center}
\end{table}

\subsection{Defining a robot with robot\_init}

The robot\_init function sets some informations such as the direction
faced by the robot at start, the position of the robot at start, and
the sprites used to draw the robot. These 3 actions use the global
map\_ variable. These three actions should be performed (or else the
program may be malfunctionning), but some other actions could be
performed such as the description and initialization of the sensors
and actuators of your robot. You could also decide to give some
extra-features to your robot such as a battery-state, this is the
right place to define it, semantically (defining your actuators,
sensors or extra-features somewhere else in the file is still
possible).

We will now see the basic actions that must be performed and some
example of what you can do.

\subsubsection{Where the robot starts}

This action is \emph{mandatory}. You have to specify where your robot
starts on the map. You have to use the following function :

\begin{lstlisting}[language=Python]
map_.setPosition(xPosition, yPosition)
\end{lstlisting}

The xPosition (resp. yPosition) corresponds to the abscissa
(resp. ordinate) where the robot will be placed at start. These
coordinates refer to the TMX map loaded. The (0,0) box is located on
the top-left of the TMX map.

You also have to specify in which direction the robot watches. Only
maps with 4 directions are currently supported. These directions are
defined by these pairs :


\begin{table}[h]
  \begin{center}
    \begin{tabular}{|c|c|}
      \hline
      Pair   & Meaning \\
      \hline
      (1,0)  & East    \\
      (0,-1) & North   \\
      (-1,0) & West    \\
      (0,1)  & South   \\
      \hline
    \end{tabular} 
    \caption{\label{tab:Pypairs} The avaible pairs to describe a
      direction}
  \end{center}
\end{table}


You have to use the following function which is \emph{mandatory} :

\begin{lstlisting}[language=Python]
map_.setPosition(x,y)
\end{lstlisting}

Where (x,y) is one of the pairs presented in the table \ref{tab:Pypairs}.

\subsubsection{What the robot looks like}

In order to represent the robot on the map, you have to give an image
uri which will reprensent it. Since your robot can face 4 different
directions, it is asked to give 4 images, one for each direction the
robot can face. However, you can put 4 times the same picture.

The order for the pictures is north, south, east, west.

You have to use the following function wich is \emph{mandatory} :

\begin{lstlisting}[language=Python]
map_.setRobotImage(northImg, southImg, eastImg, westImg)
\end{lstlisting}

Where northImg (resp. southImg, eastImg, westImg) corresponds to the
image to use when the robot faces north (resp. south, east, west).

\subsubsection{Sensors}

You have to describe the robot's sensors. A sensor is a feature that
gives the possibility to your robot to have information on its
environment. Even if you can instantiate and use a sensor from
anywhere, it is highly recommended to declare it in the robot\_init()
function (or in a subfunction called by robot\_init()).

A sensor can get an information about a tile at a fixed range, and on
a precise layer. You can find the list of the avaible layers in
section \ref{avaibleLayers} on page \pageref{avaibleLayers}.

For example, it can have access to the tile right in front of the
robot (range 1), and sense the Object layer (enums.Level.Object).

To declare a sensor, you may use the following syntax :

\begin{lstlisting}[language=Python]
  global mySensor
  mySensor = sensor(map_,enums.Level.Ground,2)
\end{lstlisting}

Where enums.Level.Object is part of the avaible layers, 2 is the range
of the sensor (it could be 1, 3, or any positive integer). In this
example, the sensor gets the tile on the Ground layer at a distance of
2 tiles ahead of the robot.

Of course, it is recommended to use better names for your
sensors. Here, we could have called it groundSensor2.

\subsubsection{Actuators}

You have to describe the robot's actuators. An actuator is a feature
that gives the possibility to your robot to modify its
environment. Even if you can instantiate and use an actuator from
anywhere, it is highly recommended to declare it in the robot\_init()
function (or in a subfunction called by robot\_inti()).

An actuator can modify a tile at a fixed range, on a precise layer and
in a definite way. You can find the list of the avaible layers in
section \ref{avaibleLayers} on page \pageref{avaibleLayers}. An actuator replaces a tile by another
tile. The new tile must be defined in the tileset of the map. Its name
is used as the description of the tile.

To declare an actuator, you may use the following syntax :

\begin{lstlisting}[language=Python]
  global myActuator
  myActuator = actuator(map_,enums.Level.Object,1, ``toto'')
\end{lstlisting}

Where enums.Level.Object is part of the avaible layers, 1 is the range
of the actuator (it could be 2, 3, or any positive integer) and toto
is the new tile that will replace the current one. In this example,
when an myActuator is activated, it replaces the tile in front of the
robot on the Object layer by a tile named ``toto'' which must me
described in the tileset used in the map.


\subsubsection{The compass}

The compass is another type of sensor wich is provided by the
application. Even if you could emulate a compass with python by
remembering in which direction you started, and updating your current
direction each time you turn, the compass sensor allows you to get
your current direction at any time.

You can define a new compass by using this syntax :

\begin{lstlisting}[language=Python]
  global compassSensor
  compassSensor = compass(map_)
\end{lstlisting}


\subsubsection{Extending your robot abilities}

Of course, you can add many features to your robot. You could decide
to extend the sensors and actuators with a battery state, and updating
it each time you use one of these.

You could also add a memory to your robot, so it can remember the
paths it took.

You are free and encouraged to extend your robot abilities so it
reflects as much as possible the abilities and constraints of a real
robot.

\subsection{Defining a strategy with strat}

As mentioned above, the editable strat function is the procedure that
the robot will follow for one iteration of the overall strategy.  That
is to say, defining the sensors, actuators of other available actions
the robot will use and how, from the tile it is currently on.

Some variables can be memorized in order to change the behavior of the
robot for a given iteration of the strategy, but every step of the
strategy must be defined by the strat function.

\subsubsection{Using the sensors and the actuators}

The sensors and actuators are called in the strategy by adding
.activate() to the method, for instance
\begin{lstlisting}[language=Python]
  groundSensor.activate()
\end{lstlisting}
will return the information given by the ground sensor.

This is and has to be the only way for the robot to get any
information about the map.

\subsubsection{Some useful functions}

Other functions than the sensors and actuators are available, but they
can't be defined as such because they interact with the robot instead
of the map. These are the basic movement functions to turn left
(turnLeft), turn right (turnRight) and move one tile ahead (walk).

From these basic functions it is possible to build new useful methods,
for instance moving two tiles ahead could be defined by two iterations
of the basic walk function.

Of course, and as it is said before, you can implement many more
extended methods this way.

\subsection{Specifying the end with isEnded}

Unless you want to create an infinite loop, your strategy has to have
an end. You can set the end of the strategy by using the ended
variable to ``True'':

\begin{lstlisting}[language=Python]
  global ended
  ended = True
\end{lstlisting}

This strategy ending flag is tested when the iteration is over, so
that the program will stop the strategy procedure when it is set to
``True''.

You may want to use this variable with a condition, so that it will
only trigger the end of the strategy on a given iteration.

\bibliographystyle{plain}
\bibliography{biblio}

\end{document}
